\section{The LSST Science Drivers }
\label{scidriv}

The LSST collaboration has identified the aforementioned four science
programs as the normative key drivers of the science
requirements\footnote{The ``normative key drivers of the science requirements'' define
the scientific product of the LSST mission which is realized during survey operations phase.
Normative key drivers of the science requirements lead in turn to actionable engineering project
requirements that must be achieved at the end of the construction phase.  Please refer to the
LSST System Requirements (LSST Document \citeds{LSE-29}) for a complete discussion of system requirements.}
for the project. Their selection was the result of discussions within the
consortium and reflects the input of
\begin{itemize}
\item the four National Research Council studies\footnote{
  Astronomy and Astrophysics in the New Millennium, NAS 2001 \citep{NAP9839};
  Connecting Quarks with the Cosmos: Eleven Science Questions for the New Century, NAS 2003 \citep{NAP10079};
  New Frontiers in the Solar System: An Integrated Exploration Strategy, NAS 2003 \citep{NAP10432};
  New Worlds and New Horizons in Astronomy and Astrophysics, NAS 2010 \citep{NAP12982}.
} that have endorsed the LSST,
\item the report of the LSST Science Working Group (SWG), an independent
      committee formed by NOAO to represent community interests
\item the scientific interests of the partners in the LSSTC
\item the physics and astrophysics community.
\end{itemize}

The SWG report \citedsp{Document-26952} (also known as the Strauss
report), the LSST NSF MREFC proposal\footnote{Available (to LSST) as
\citeds{Document-10549}},
the LSST Dark Energy Task Force report \citep{2006astro.ph..9591A}, and the LSST
Science Book \citep{2009arXiv0912.0201L} should be
consulted for a more detailed discussion of the major scientific advances
that can be expected from the construction of a wide-field telescope that
is dedicated to repeated, deep, multi-color imaging of the sky.

For each of the four primary science drivers selected by the LSST collaboration,
this Section briefly describes the science goals and the most challenging
requirements for the LSST system that are derived from those goals (separate
documents will deal with more detailed requirements\footnote{The two
high-level documents derived from this document and other constraints are the
LSST System Requirements Document \citedsp{LSE-29} and the LSST Observatory System
Specifications Document \citedsp{LSE-30}. For an overview of high-level LSST documents,
please see the LSST Document Tree \citedsp{LSE-39}.}).
Tables are also provided in the subsequent Section, that
integrates the detailed requirements of these four programs. If these
requirements are met by the LSST -- and indications of the preliminary
engineering studies undertaken to date indicate that they can be -- then
the LSST will not only enable all four of these major scientific
initiatives but will also make it possible to pursue many other research
programs. Some examples are described in the LSST Science Book,
but the long-lived
data archives of the LSST will have the astrometric and photometric
precision needed to support entirely new research directions which will
inevitably develop during the next several decades.



\subsection{Constraining Dark Energy and Dark Matter \label{sec:DE}}

Driven by observations, current models of cosmology require the existence
of both dark matter and dark energy (DE). One of the primary challenges for
fundamental physics is to understand these two major components of the
universe. In addition to making a unique map of dark matter structure over
half the sky, LSST will probe dark energy in multiple ways, providing cross
checks and removal of important degeneracies.
The primary DE science drivers for LSST come from a suite of two
and three point cosmic shear tomography analyses coupled with galaxy power
spectrum and baryon acoustic oscillation (BAO) data, as well as from the
use of supernovae as standard candles. Due to its wide area coverage, LSST
will be uniquely capable of measuring 7 parameters related to DE:
the lowest 6 eigenmodes of the DE equation of state vs. redshift, $w(z)$, and
any directional dependence.
Combining these probes, LSST will measure the comoving distance as a function
of redshift in the redshift range 0.3--3.0 with an accuracy of 1-2\%, and
separately the growth of cosmic mass structure.
A sample of about four billion galaxies with sufficiently accurate photometric
redshifts is required. In order to achieve this comoving distance accuracy,
the photometric redshifts requirements for this $i<25$
flux-limited galaxy sample are i) the rms ($\sigma$) for error in $(1+z)$ must be
smaller than 0.02, ii) the fraction of ``catastrophic'' outliers
\newtext{(defined as those with errors exceeding the larger of 0.06 and $3\sigma$)}
must be below 10\%, and iii) the bias must be below 0.003. These requirements are
primary drivers for the photometric depth of the main LSST survey. In addition,
methods for rejecting the majority of those outliers, and for characterizing their
effects on the sample, must be developed. The calibration of photometric redshifts
and their errors can be a combination of correlation with bright spectroscopic samples
and spot-checks with many-band photometric redshift samples. Combining BAO with
weak lensing of galaxies can significantly reduce sensitivity to bias systematics.

DE exerts its largest effects at moderate
redshift; LSST's redshift coverage will bracket the epoch at which DE began
to dominate the cosmic expansion. When combined with Planck CMB data, the
LSST data will sharply test models of DE, whether due to new
gravitational physics, vacuum energy, or other causes.




\subsubsection{Weak Lensing Studies}
\label{WLstudies}

Weak lensing (WL) techniques can be used to map the distribution of mass as a function
of redshift and thereby trace the history of both the expansion of the universe and the
growth of structure (see Chapter 14 in the LSST Science Book). These investigations use
common deep wide-area multi-color imaging with stringent
requirements for the shear systematics in at least two bands and photometry in all bands.
These requirements are covered in more detail in the LSST DETF report and references therein.

The shear systematic errors can be mostly corrected by use of foreground stars.
The spatially varying PSF within each exposure must be mapped, fit, and corrected. The precision of this
correction depends on how many stars are available, and thus depends on the angular scale.
The overall scale of the combined errors is set by the requirement of distinguishing models
of the origin of DE: unique sensitivity to the cosmic shear power spectrum from arcminute
to 100 degree scales and wide redshift range, the ability to probe at least six DE
eigenfunctions, and any variation over the sky. This leads to an \'{e}tendue requirement
for \textit{areal coverage times depth} (several billion source galaxies to $z$=3), as well
as photometric precision and wide angular coverage ($>$ 90 deg).

The power of the LSST relative to existing weak lensing surveys derives from its
ability to survey much larger areas of the sky to faint limiting
surface brightness while maintaining exquisite control of systematic errors in the
galaxy shapes. Characterizing dark energy places particularly strong
requirements on the total area of sky covered, the depth of the stacked image, the
number of revisits to each field, the ellipticity and sampling of the point spread
function (PSF), and the choice of filters, which must be suited to allow accurate
photometric redshifts to be measured. At least six bands are required. Photometric precision of at
least 1\% is required, as well as quite accurate calibration of photometric redshifts over the
redshift interval 0.3 -- 3.

The scale of residual shear errors should be set by the statistical error floors on the
coadded data, not systematics. The two components of statistical shear errors vary oppositely
with angular scale. On small angular scales ($<$ few arcminutes) the source galaxy
shear error is dominated by the random ``shot'' noise of the galaxy intrinsic
ellipticities (about $e$=0.3 rms per galaxy) and the finite areal density of source
galaxies. On large angular scales the source shear error is dominated by large scale
structure cosmic variance. The cross-over point varies with source redshift. For
all redshifts in projection,
the two errors sum to nearly a constant statistical shear power of $3\times 10^{-7}$,
or a source rms residual ellipticity of 0.001, over the range of angular scales for LSST WL science.
The residual shear power
systematics at all angular scales (\textit{after PSF corrections}) must be less than 30\% of
the statistical shear power,  including correlations between angle bins. To achieve
this goal, the residual shear power systematics (after corrections) must be below the
statistical errors by a factor of $\sim$3.
While the statistical error is uncorrelated with angular scale (as source galaxies are randomly
oriented), systematic errors are typically correlated. Statistical errors are reduced when
averaged over many exposures and a broad angular band, but systematics do not average
down unless they are chopped or vary stochastically from exposure to exposure due to seeing.
Thus there are two limiting angular regimes with different methods for reductions of systematics:
(1) arcminute scale systematics in the residual PSF ellipticity correlation average down like the
number of exposures [further tests are needed for large N], and
(2) degree scale residual systematics can be reduced via chopping by dithering and rotating.
In both cases further tests are needed to make sure residuals continue to average down to the
needed level.



\subsubsection{Supernovae}

Supernovae (SN) provided the first evidence that the expansion of the
universe is accelerating. LSST will be a powerful SN factory
(see Chapter 11 in the LSST Science Book). Operating in
a standard mode of repeated scans of the sky with images taken every few
days and with exposures of 30 seconds, LSST will discover
of the order $10^5$ Type Ia
SN annually. Their mean redshift will be $z\sim$0.45 with a maximum
redshift of $\sim$0.7. These data, when combined with priors from other
experiments, can constrain the lowest eigenmode of $w$ (\ie the mean
value) in the nearby universe to 1\% (limited by systematics),
and given the dense sampling on
the sky, can be used to search for any dependence of $w$ on direction,
which would be an indicator of new physics.  Some SN will be located in the
same direction as foreground galaxy clusters; a measurement of the
magnification of the SN will make it possible to model the cluster mass
distribution. Core-collapse SN will provide estimates of the star formation
rate during the epoch when star formation was changing very rapidly.
Longer exposures (10-20 minutes/band) of a small area of the sky could
extend the discovery of SN to a mean redshift of 0.7 with some objects
beyond $z\sim$1.  The added statistical leverage on the
``pre-acceleration'' era will narrow the confidence interval on both $w$
and its derivative with redshift.

Spectroscopic follow-up for so many SNe will be impossible. Exploitation of
the data from the LSST will require light-curves which are well-sampled both
in brightness and color as a function of time. This is essential to the
search for systematic differences in supernova populations which may
masquerade as cosmological effects as well as for determining photometric
redshifts from the supernovae themselves; the development of techniques for
determining photometric redshifts from supernova light-curves is currently
being pursued by several community groups. Good image quality is required
to separate SNe photometrically from their host galaxies. Observations in
five photometric bands will be necessary to ensure that, for any given
supernova, light-curves in four  bands will be obtained (due to the spread in
redshift). Absolute band-to-band photometric calibration to 1\% is adequate, but
the importance of K-corrections to supernova cosmology implies that the
calibration of the relative offsets in zero points between filters remains
a serious issue, as
is stability of the response functions, especially near the edges of
bandpasses where the strong emission and absorption features from supernovae
makes this more of a problem than for stellar spectra.

\subsection{Taking an Inventory of the Solar System}

LSST will provide data for millions of small bodies in our Solar System. Previous studies
of these objects have led to dramatic changes in our understanding of the process of
planet formation and evolution, and the relationship between our Solar System and other
systems. These small bodies also serve as large populations of ``test particles'', recording
the dynamical history of the giant planets, revealing the nature of the Solar System impactor
population over time, and illustrating the size distributions of planetesimals, which were the
building blocks of planets  (see Chapter 5 in the LSST Science Book).

The Earth orbits within a swarm of asteroids; some small number of these
objects will ultimately strike the Earth's surface. The U.S. Congress has
mandated that by the year 2008, 90\% of the near-Earth asteroids (NEAs)
with diameters greater than 1 km be discovered and their orbits
determined. Impacts of NEAs of this size have the potential to change the
Earth's climate and cause mass extinctions, such as the one credited with
killing the dinosaurs. A NASA report published in 2003 estimates
conservatively that with current search techniques, about 70\% of the NEAs
with diameters larger than 1 km will have been be cataloged by 2008. This same report
quantifies the risk of impacts by smaller bodies, which have the potential
of causing significant ground damage, and recommends reduction of the residual
hazard by another order of magnitude  as a reasonable next
goal. Achieving this goal would require discovery of about 90\% of the
potentially hazardous asteroids (PHAs) down to diameters of about 140
m. While it is unlikely that any other currently planned facility could achieve
this goal within a decade or two, modeling suggests that the LSST is capable
of finding 84\% of the PHAs with diameters larger than 140 m within
ten years.

The search for PHAs puts strong constraints on the cadence of observations,
requiring closely spaced pairs of observations two or preferably three
times per lunation in order to link observations unambiguously and derive
orbits. Individual exposures should be shorter than about 1 minute each to
minimize the effects of trailing for the majority of moving
objects. Because of the faintness and the large number of PHAs and other
asteroids that will be detected, LSST must provide the follow-up required
to derive orbits rather than relying, as current surveys do, on separate
telescopes. The observations should be obtained within $\pm15$ degrees of
the Ecliptic.  The images should be well sampled to enable accurate
astrometry, with absolute accuracy not worse than 0.1 arcsec for sources
detected with the signal-to-noise ratio $SNR>10$. There are no
special requirements on filters, although bands such as V and R that offer
the greatest sensitivity are preferable.  The images should reach a depth
of at least 24.5 (5$\sigma$ for point sources) in the $r$ band in order to
probe the $\sim0.1$ km size range at main-belt distances. Based on recent
photometric measurements of asteroids by the Sloan Digital Sky Survey, the
photometry should be better than 1-2\% to allow for color-based taxonomic
classification.

The LSST can also make a major contribution to mapping Kuiper Belt Objects
(KBOs).  The orbits of KBOs provide a fossil record of the early history of
the solar system; their eccentricities and inclinations contain clues to
past perturbations by giant planets. The sizes of the KBOs hold clues to
the accretion events that formed them and to their subsequent evolution
through collisional grinding, etc. The compositions of KBOs are not
identical and are correlated with their dynamical state; the reasons for
these differences are not known. Light curves can be used to constrain the
angular momentum distribution and internal strengths of the bodies. A more
complete sample of KBOs and determination of their properties can assist
with selecting targets for future NASA missions. The survey for PHAs can
simultaneously provide the joint color-magnitude-orbital distribution for
all bright ($r<24$) KBOs. The 100 or so observations obtained for each
bright KBO can be searched for brightness variations, but modeling will be
required to determine how well periods can be extracted from observations
made at random times. At the very least, it will be possible to determine
amplitudes for many thousands of KBOs, and periods can likely be derived
for many of them.

Long exposures would be required to push the detection of KBOs to smaller
sizes and reach the erosion-dominated regime in order to study the
collisional history of various types of KBOs. KBO science would be greatly
amplified if a small fraction of the observing time were devoted to
hour-long observations in the ecliptic. This same mode of observation may
have applications to the study of variable and transient objects. Apart
from exposure time limits, the requirements for the KBO
science are similar to the requirements for the detection and
orbital determination for other Solar System bodies.



\subsection{Exploring the Transient Optical Sky}
\label{sec:transients}

The LSST will open a new window on the variable sky (see Chapter 8 in
the LSST Science Book). Recent surveys have
shown the power of variability for studying gravitational lensing,
searching for supernovae, determining the physical properties of gamma-ray
burst sources, etc. The LSST, with its repeated, wide-area coverage to deep
limiting magnitudes will enable the discovery and analysis of rare and
exotic objects such as neutron star and black hole binaries; gamma-ray
bursts and X-ray flashes, at least some of which apparently mark the deaths
of massive stars; AGNs and blazars; and very possibly new classes of
transients, such as binary mergers and stellar disruptions by black holes.
It is likely that the LSST will detect
numerous microlensing events in the Local Group and perhaps beyond.  The
LSST would provide alerts for concerted monitoring of these events, and
open the possibility of discovering planets and obtaining spectra of lensed
stars in distant galaxies as well as our own.  LSST can also provide
multi-wavelength monitoring over time of objects discovered by the
Fermi Gamma-ray Space Telescope (formerly GLAST)
and the Energetic X-ray Imaging Survey Telescope (EXIST). With its large aperture, the LSST is well
suited to conducting a Deep Supernova Search in selected areas.  LSST will
also provide a powerful new capability for monitoring periodic variables,
such as RR Lyrae stars, which can be used to map the Galactic halo and
intergalactic space to distances exceeding 400 kpc. Since LSST extends
time-volume space a thousand times over current surveys, the most
interesting science may well be the discovery of new classes of objects.

Exploiting the capabilities of LSST for time domain science requires large
area coverage to enhance the probability of detecting rare events; time
coverage, since light curves are necessary to distinguish certain types of
variables and in some cases infer their properties (\eg determining the
intrinsic luminosity of supernovae Type Ia depends on measurements of their
rate of decline); accurate color information to assist with the
classification of variable objects; good image quality to enable
differencing of images, especially in crowded fields; and rapid data
reduction and classification in order to flag interesting objects for
spectroscopic and other follow up with separate facilities. Time scales
ranging from $\sim$1 min (to constrain the properties of fast faint
transients such as those recently discovered by the Deep Lens Survey) to
$\sim$10 years (to study long-period variables and quasars) should be
probed over a significant fraction of the sky. It should be possible to
measure colors of fast transients on timescales of a few minutes,
and to reach $r \sim 24$ in individual visits. Fast reporting of likely transients
to the community is required in order to facilitate followup observations.



\subsection{Mapping the Milky Way \label{sec:MW}}


The LSST is ideally suited to answering two basic questions about the Milky
Way Galaxy: What is the structure and accretion history of the Milky Way?
What are the fundamental properties of all the stars within 300 pc of the
Sun? (see Chapters 6 and 7 in the LSST Science Book).

Standard models posit that galaxies form from seeds planted by the Big Bang
with accretion over time playing a significant role in determining their
structure.  Detailed study of the Milky Way can provide rigorous tests of
these ideas, and the LSST will be able to map the 3-D shape and extent of
the halo of our Galaxy.  Specifically, the LSST will detect F turn-off
stars to distances of 200 kpc; isolate stellar populations according to
color; and determine halo kinematics through measurement of proper motions
at distances exceeding 10 kpc. The LSST dataset can be used to identify
streams of stars in the halo that are thought to provide a fossil record of
discrete accretion events. The LSST in its standard surveying mode will be
able to detect RR Lyrae variables and classical novae at a distance of 400
kpc and hence can explore the extent and structure of our own halo out to
half the distance to the Andromeda Galaxy. The proper motions and
photometric parallaxes for these stars can be used to characterize the
properties of the dark matter halo in which the Milky Way is embedded.
The LSST will survey a significant fraction of the Galactic plane,
including the Galactic center, and will obtain unprecedented data for
studies of star-forming regions.

Is our solar system with its family of planets unique? Or are there many
more that contain Earth-like planets within the so-called habitable zone?
How do solar systems form? Detailed exploration of our local neighborhood
is key to answering these questions.  The LSST will obtain better than
3$\sigma$ parallax measurements of hydrogen-burning stars to a distance of
300 pc and of brown dwarfs to tens of parsecs. These measurements will
provide basic information on candidate stars that merit further study in
the search for companions, including planets.  Residuals from the fits for
position, proper motions, and parallax will be searched for the signature
of Keplerian motion to identify stars and brown dwarfs with companions and
provide fundamental estimates of the mass of the primaries. LSST data will
be used to determine the initial mass functions for low-mass stars and
sub-stellar mass objects and to test models of brown dwarf structure. The
age of the Galactic disk can be inferred from white dwarf cooling curves.

Key requirements for mapping the Galaxy are large area coverage; excellent
image quality to maximize the accuracy of the photometry and astrometry,
especially in crowded fields; photometric precision of at least 1\% to
separate main sequence and giant stars; stringent astrometric accuracy to
enable parallax and proper motion measurements; and dynamic range that
allows measurement of astrometric standards at least as bright as r =
15. In order to probe the halo out to distances of 100 kpc using large numbers of
main sequence stars, the total depth ($5\sigma$ for unresolved sources)
has to reach $r\sim27$ (assuming 5\% photometry in the $r$ band at $r=25.5$).
To study the metallicity distribution of stars in the Sgr tidal stream and other halo
substructures at distances out to at least $\sim$40 kpc, the coadded depth in
the $u$ band has to deliver 5\% photometry at $u\sim24.5$.
In order to constrain tangential velocity at a distance of 10 kpc to within 10 km/s
with the most luminous main-sequence stars (low-metallicity blue turn-off stars with $M_r=5.5$),
the proper motion accuracy has to be at least 0.2 mas/yr at $r=20.5$ (1$\sigma$ per coordinate).
The same requirement follows from the decision to obtain the same proper motion accuracy
as Gaia at its faint end ($r\sim20$). The LSST will then represent an ``extension'' of Gaia
astrometric measurements to 4 magnitudes greater depth. In order to produce
a complete sample of the solar neighborhood stars out to a distance of 300
pc (the thin disk scale height), with 3$\sigma$ or better geometric
distances, parallax measurements accurate to 1 mas (1$\sigma$) are required
for stars with $M_r=15$.
To obtain 3$\sigma$ or better geometric distances for T9/Y0 brown dwarfs with
$z-y$ colors measured with 10$\sigma$ or better precision (in coadded data),
parallax  measurements for sources detected only in $y$ band visits at 10$\sigma$
significance must have an accuracy of 6 mas (1$\sigma$).

In summary, these requirements imply that the LSST will enable studies of the
distribution of numerous main-sequence stars beyond the presumed edge of
the Galaxy's halo, of their metallicity distribution throughout most of the
halo, and of their kinematics beyond the thick disk/halo boundary, and will
obtain direct distance measurements below the hydrogen-burning limit for a
representative thin-disk sample.

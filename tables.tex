\newpage
\section{Detailed Description of Science Requirements }

The purpose of this Section is to lay out a common set of science
requirements necessary to achieving a set of concrete scientific measurements,
of specified accuracy, in the four main science areas described above.  It
will serve as the primary starting point for deriving engineering project
requirements to be placed upon the various technical subsystems that
comprise the LSST. Note that some of these requirements are not fully
independent of the existing baseline design (see Appendix A), and of
realities such as seeing and sky brightness distribution at the selected
site (Cerro Pach\'{o}n in Chile). While different science programs
require broadly consistent datasets, the adopted values represent the
most stringent requirements from the previous section.


\subsection{The Definitions of Specified Parameters }

For each quantity specifying a requirement, we identify two values: a
\textit{minimum specification}, and a \textit{design specification.}

\textit{The minimum specification} shall represent the minimum capability
or accuracy required of the system in order to achieve its scientific
aims. If the design analysis clearly demonstrates that a minimum
specification requirement cannot be met, the Science Council will
reevaluate the science drivers that led to the specification, estimate
the scientific impact of the failure to meet the specification, and
report the findings to the Project Director and Project Manager.

\textit{The design specification} represents the system design point and will
be used as the basis for developing engineering tolerances.  At the time
this document is written, we believe that the design specification should
be achievable in the context of the existing baseline concept for the LSST.
However, as development proceeds, it is conceivable that there may be some
change in capability away from these values.

In some cases, \textit{stretch goals} are specified. These are desirable
system capabilities which will enhance scientific return if they can be
achieved. Stretch goals are to be pursued if they do not significantly
increase cost, schedule or risk. To avoid complication and ambiguity, we do
not list these in every instance; it remains understood that wherever
improved capability is easily achievable, it should be pursued.  Situations
where enhanced capability beyond the design specification compromises cost,
schedule, or other system parameters must be evaluated on a case-by-case
basis to decide whether they make sense in the context of the whole system.
In addition to numerical requirements, a brief reference to the science
program that places the strongest constraints is also provided.


\subsection{Distinction between Single Image Specifications and the
                        Full Survey Performance \label{sec:defs}}


Detailed simulations show that the LSST will be capable of obtaining over
200,000 10 deg$^2$ images per year, assuming 30 sec total exposure per
image and realistic observing conditions for Cerro Pach\'{o}n. For each of
these images, a decision involving at least three free
parameters (position on the sky and filter; assuming fixed exposure time
and position angle of the field of view) must be made. With a
simplifying assumption of only 2,000 allowed sky positions (\ie a
fixed grid of 10 deg$^2$ field centers tiling an area of 20,000 deg$^2$)
and 5 filters, there will be over 10$^{10}$ different ways to execute the
LSST observations over its projected 10-year lifetime. Hence, the
optimization of LSST observing strategies is a formidable problem that
requires significant additional analysis. For this reason, only weak
constraints for observing cadence are listed here (though integral
quantities such as total depth and sky coverage are specified).
The required properties of individual images (also known as
\textit{visits}, consisting of two co-added, back-to-back exposures), however,
are specified
in detail because they directly constrain the capabilities of the hardware
and software systems. The detailed error budget distribution between the hardware
and software systems is not considered here and will be addressed in a
separate documents (the LSST System Requirements Document and the LSST System
Architecture Document).
As a general principle, the measurement errors for fundamental quantities,
such as astrometry, photometry and image size, should not be dominated
by algorithmic performance.



%%%%%%%%%%%%%%%%%%%%%%%%%%%%%%%%%%%%%%%%%%%%%%%%%%%%%%%%%%%%%%%%%%%%%%
%%%%%%%%%%%%%%%%%%%%%%%%%%%%%%%%%%%%%%%%%%%%%%%%%%%%%%%%%%%%%%%%%%%%%%
%%%%%%%%%%%%%%%%%%%%%%%%%%%%%%%%%%%%%%%%%%%%%%%%%%%%%%%%%%%%%%%%%%%%%%

\subsection{              Single Image Specifications              }
\label{singleImageSpecs}


The fundamental image properties specified in this section are
\begin{itemize}
\item Bandpass characteristics
\item Image depth (attained magnitude at some fiducial signal-to-noise ratio)
\item Image quality (size and ellipticity)
\item Astrometric accuracy
\item Photometric accuracy
\end{itemize}

There are several factors that increase the complexity of these
specifications. Many of the image properties depend on quantities such
as zenith angle (airmass), wavelength, sky brightness, relative positions
on the sensor and within the field of view, and attained signal-to-noise
ratio.  Most of these quantities are actually distributions, and can be
specified by a single number only in special cases, such as that of a
perfect Gaussian distribution (with zero mean).

We address these complexities as follows. For quantities with strong
wavelength dependence, requirements are specified in each band. Where
relevant, fiducial seeing and airmass are specified. For quantities with a
strong dependence on the signal-to-noise ratio (SNR), requirements are
specified at the \textit{bright end}, defined here as the magnitude range
between 1 mag and 4 mag fainter than the saturation limit (full well) in a given
bandpass.  Assuming that the faint end of this range corresponds to $r=20$,
and that 5$\sigma$ depth is achieved at $r=24.5$, the photon statistics
limits on photometric and astrometric accuracy ($SNR\sim200$) are 5 millimag
and 4 milliarcsec for a fiducial delivered seeing of 0.7 arcsec. Both of these
limits are sufficiently small as to allow the required overall photometric
and astrometric accuracy described below (which include effects such as sky
brightness and instrumental noise, as well as various calibration
uncertainties).  About 1\% of all the sources detected in a typical LSST
image will be brighter than $r=20$. At this magnitude, the surface
densities of galaxies and high-Galactic-latitude stars are similar: about
1000 per square degree (implying a typical nearest-neighbor distance for
stars of the order of 1 arcmin).

We define ``FWHM'' as \textit{the full width at half maximum}, and ``rms'' as
\textit{the root-mean-square scatter}. For a one-dimensional Gaussian distribution,
FWHM = 2.35\,rms.


%%%%%%%%%%%%%%
\subsubsection{Filter Set Characteristics}

The filter complement (Table~\ref{Tfilters}) is modeled after the Sloan
Digital Sky Survey (SDSS) system because it has demonstrated success in a
wide variety of applications such as photometric redshifts of galaxies,
separation of stellar populations, and photometric selection of quasars.

\begin{table}[h!]
\begin{tabular}{|r|r|r|r|}
\hline
Quantity               & Design Spec & Minimum Spec & Stretch Goal      \\
\hline
 Filter complement     &   ugrizy      &   ugrizy       &   ubgrizy \\
\hline
\end{tabular}
\caption{The filter complement.}
\label{Tfilters}
\end{table}

\vskip -0.1in
The extension of the SDSS system to longer wavelengths (the $y$ band at
$\sim$1 $\mu$m) is
mandated by the increased effective redshift range achievable with the LSST
due to deeper imaging, and the desire to study regions of the Galaxy that
are obscured by interstellar dust.
The optimal wavelength range for the $y$ band is
still under investigation. A narrow, blue $b$ filter may increase the
photometric redshift accuracy, but the quantitative effects of its addition
are also under investigation. The addition of the $u$ band will improve the
robustness of photometric redshifts of galaxies and stellar population
separation, will enable quasar color selection and stellar metallicity
estimates, and will provide significant additional sensitivity to star
formation histories of detected galaxies (\eg GALEX bands are redshifted
to the $u$ band for galaxies at redshifts of about 1, close to the median
redshift for galaxies detectable in deep LSST images). The current design of
the bandpasses is illustrated in Appendix C.



\paragraph{The Number of Filters Used in a Night\\}

The number of filters,
\param{Nfilters}{The number of filters that can be housed simultaneously
within the camera.},
to be used on the same night is
equivalent to the number of filters that can be simultaneously housed
within the camera. It is assumed that any other filter from the filter
complement can be inserted during the daytime, even on the shortest
winter days.

\textbf{Specification:} The number of filters available at any given time is
specified as Nfilters (Table~\ref{Tfilterchanges}).

\begin{table}[bh!]
\begin{tabular}{|l|r|r|r|}
\hline
Quantity        & Design Spec & Minimum Spec & Stretch Goal      \\
\hline
 Nfilters       &      5      &      3       &       6           \\
\hline
\end{tabular}
\caption{The number of filters that can be housed simultaneously within the
camera.}
\label{Tfilterchanges}
\end{table}


\newpage
\textbf{Specification:} The maximum time allowed to switch filters already
present inside the camera is specified as
\param{TFmax}{The maximum time allowed to switch filters already
present inside the camera (minutes).}
(Table~\ref{Tfilterswitch}).

\begin{table}[h]
\begin{tabular}{|r|r|r|r|}
\hline
     Quantity     & Design Spec & Minimum Spec  & Stretch Goal      \\
\hline
 TFmax (min)      &      2      &      10       &       1           \\
\hline
\end{tabular}
\caption{The maximum time, in minutes, allowed to switch filters already
present inside the camera.}
\label{Tfilterswitch}
\end{table}

The ability to rapidly switch active filters will allow more useful color
measurements of fast transients.


\paragraph{Filter Out-of-Band Constraints\\}

\textbf{Specification:} Beyond the wavelengths more than one FWHM from the filter
center wavelength (including hardware and atmosphere), the mean transmission
in any 10nm interval must be less than
\param{Fleak}{The maximum permitted out-of-band flux for filters
(defined as flux, normalized by the peak value, in any 10nm interval
at wavelengths beyond those where the transmission curve decreases to
below 0.1\% of its peak value for the first time) (\%).}
\% of the peak value, and the integrated transmission at those wavelengths
must be below
\param{FleakTot}{The maximum integrated out-of-band filter transmission at
all wavelengths beyond those where the transmission curve decreases to
below 0.1\% of its peak value for the first time (\%).}
\% of the overall transmission (Table~\ref{Tleaks}).

\begin{table}[h]
\begin{tabular}{|l|r|r|r|}
\hline
Quantity    & Design Spec & Minimum Spec & Stretch Goal        \\
\hline
 Fleak (\%)      &      0.01      &      0.02   &       0.003     \\
 FleakTot (\%)   &      0.05      &      0.1    &       0.02      \\
\hline
\end{tabular}
\caption{Filter Out-of-Band Constraints (transmission in \% of the peak
value in any 10nm interval beyond one FWHM of the central wavelength,
Fleak, and total transmission out of band, FleakTot).}
\label{Tleaks}
\end{table}

This requirement assures reasonable photometric accuracy for objects with
extreme colors. For example, for a source with the color $u-i=5$, the
effect of a $u$ band red leak confined to the $i$ band (\eg see Appendix
C) is limited to a 0.05 mag bias in the $u$ band measurement ($<$0.01 mag
for sources bluer than $u-i=3$).

The temporal change of bandpasses (due to aging of the filters, changes in
reflectivity of coatings, \textit{etc.}) must be sufficiently small to enable
the required photometric calibration accuracy (specified below, see \S
\ref{photoacc}).



%%%%%%%%%%%%%%
\subsubsection{Image Depth and the Minimal Exposure Time}

An \textit{exposure} means a single readout of the camera (one of the two
back-to-back exposures designed for cosmic ray rejection that together
represent a \textit{visit} to a target field).  Image properties, such as
depth (attained magnitude for point sources at some fiducial SNR, here
taken to be 5), \textbf{are defined per visit} (not per exposure), and assume optimal
count extraction algorithms (e.g. point-spread-function magnitudes). The image
depth depends on total exposure time, bandpass, delivered image quality
(dominated by atmospheric seeing as per image quality requirement), the sky
brightness and its spatial structure, and the system efficiency. For a
given exposure time, fiducial seeing, and sky brightness, the required
image depth is an indirect constraint on the system's efficiency (assuming
a fixed effective primary mirror diameter).


%%%%%%%%%%%%%%
\paragraph{The overall image depth distribution\\}
\label{singleimagedepth}


\textbf{Specification:} The distribution of the 5$\sigma$ (SNR=5) detection
depth for point sources for all the exposures in the $r$ band will have a
median not brighter than
\param{D1}{The brightest median 5$\sigma$ (SNR=5) detection
depth for point sources for all exposures in a given band (mag).}
mag, and no more than
\param{DF1}{The fraction of images with 5$\sigma$ depth brighter than parameter Z1 (\%).}
\% of
images will have a 5$\sigma$ depth brighter than
\param{Z1}{DF1 \% of images will have a depth brighter than this value (used
to describe the tail of the distribution) (mag) (see DF1).}
mag. The implication
of many exposures only formally violates the paradigm of a single image
specification in this section; this requirement can be understood as a
probability distribution for the attained depth (DF1 is the fraction
not of all exposures, but of those in the $r$ band in good seeing on photometric
dark nights and close to the zenith, corrected to the fiducial parameters
listed in Table~\ref{ImageDtable}).

\begin{table}[h]
\label{tsi}
\begin{tabular}{|l|r|r|r|}
\hline
Quantity        & Design Spec & Minimum Spec & Stretch Goal   \\
\hline
      D1 (mag)  &    24.7     &     24.3    &     24.8        \\
      DF1 (\%)  &    10       &      20     &       5         \\
      Z1  (\%)  &    24.4     &     24.0    &     24.6        \\
\hline
\end{tabular}
\caption{Single image depth in the $r$ band (SNR=5 for point sources).
The D1 and Z1 values are expressed on the AB magnitude scale
and assume a source with spectral energy distribution $F_\nu$=constant,
fiducial seeing of 0.7 arcsec (FWHM), fiducial dark sky brightness of
21 mag/arcsec$^2$, airmass of 1.0, and a total exposure time of 30 sec.
The sky brightness is a conservative estimate corresponding
to solar maximum. Solar minimum value may be 0.3-0.4 mag fainter, resulting
in $\sim$0.2 mag deeper data. On the other hand, about  $\sim$0.2 mag loss
of depth is expected for data obtained at airmass of 1.4 (mostly due to
seeing degradation). }
\label{ImageDtable}
\end{table}

For a given exposure time and observing conditions, the required depths
primarily constrain the effective primary mirror diameter and overall
(hardware $+$ atmosphere) system throughput. The chosen exposure time per visit
(2$\times$15 sec) is a result of the survey optimization and satisfies
both the required final coadded depth, single visit depth, and the revisit
time if the effective primary mirror diameter is 6.5m.
The single visit depth is driven by transient sources and motion measurements
(for both Solar System objects and stellar proper motions) and the coadded
depth is driven by the required number of galaxies for cosmological studies
(see \S~\ref{VvsB}).



\paragraph{The variation of the image depth (throughput) with bandpass\\}

\textbf{Specification:} The median 5$\sigma$ (SNR=5) detection
depth for point sources in a given band will not be brighter than
\param{DB1}{The median 5$\sigma$ (SNR=5) detection
depth for point sources for all exposures in a given band (mag).}
mag (Table~\ref{ImageDBtable}).


\begin{table}[h]
\begin{tabular}{|l|r|r|r|r|r|r|}
\hline
 Design spec.  &       u   &     g    &      r     &   i   &   z   &  y   \\
\hline
 DB1 (mag)  &    23.9   &    25.0  &     24.7   &  24.0 &  23.3 & 22.1 \\
%\hline
%DT1 (\%)   &     15    &     50   &      70    &   70  &   60  &  25  \\
\hline
\hline
 Minim. spec.  &       u   &     g    &      r     &   i   &   z   &  y   \\
\hline
%\B{OLD DB1 (mag)}  &    23.7   &    24.8  &     24.5   &  23.8 &  23.1 & 21.9 \\
DB1 (mag)  &    23.4   &    24.6  &     24.3   &  23.6 &  22.9 & 21.7 \\
%\hline
% DT1 (\%)   &     10    &     35   &      50    &   50  &   40  &  20  \\
\hline
\hline
 Stretch goal  &       u   &     g    &      r     &   i   &   z   &  y   \\
\hline
 DB1 (mag)     &    24.0   &    25.1  &     24.8   &  24.1 &  23.4 & 22.2 \\
%\hline
% DT1 (\%)      &     20    &     65   &     85     &   85  &   75  &  30  \\
\hline
\end{tabular}
\caption{Specifications for the single visit depth (DB1) as a function
of bandpass (the $r$-band value is identical to that listed in
Table~\ref{ImageDtable}), assuming 2$\times$15 sec exposures,
a source with spectral energy distribution $F_\nu$=constant,
airmass of 1.0, the $r$-band fiducial seeing of 0.7 arcsec (FWHM), and
the $r$-band sky brightness of 21 mag/arcsec$^2$
(both seeing and sky brightness are converted to the appropriate band using standard
expressions, as implemented in the LSST Exposure Time
Calculator).}
\label{ImageDBtable}
\end{table}

The science drivers described in \S \ref{scidriv} and
the corresponding image depths listed in Table~\ref{ImageDBtable}
have motivated the baseline design parameters listed in Appendix A.
The assumed bandpasses are illustrated in Appendix C. Note that there is no
requirement that exposure time be same for all the bands.

If delivered system throughput would result in brighter limiting magnitudes
for the nominal conditions, the required DB1 values could be maintained by
increasing the integration time. For example, to account for 0.2 mag of loss
in limiting depth (equivalent to about 30\% loss of throughput), the required
additional survey time is about half a year for the $u$ and $g$ bands, and about
1 year for other bands, for a 10-year survey (assuming the time allocation per band
discussed in \S~\ref{VvsB}). However, while such strategy would deliver the
required coadded survey depth, it would have a negative impact on the single
visit depth and time sampling frequency. The system throughput should never
drop below values needed to meet minimum specifications for image depths
listed in Table~\ref{ImageDBtable}.

\paragraph{The variation of the image depth over the field of view\\}

\textbf{Specification:} For an image representative of the median depth (\ie
 with the 5$\sigma$ detection depth of D1 mag), the depth distribution
over individual devices will have no more than
\param{DF2}{The maximum fraction of images Z2 mag brighter than the
median depth over individual devices (\%).}
\% of the sample brighter by more than
\param{Z2}{DF2 \% of images on different devices will be this much brighter
than the median depth (mag) (see DF2).}
mag than the median depth (Table~\ref{depthVarFOV}).

\begin{table}[h]
\begin{tabular}{|l|r|r|r|}
\hline
Quantity   & Design Spec & Minimum Spec & Stretch Goal       \\
\hline
      DF2 (\%) &    15       &      20     &      10         \\
      Z2 (mag) &    0.2      &      0.4    &     0.2         \\
\hline
\end{tabular}
\caption{Image depth variation over the field of view. These values
apply to all bands.}
\label{depthVarFOV}
\end{table}

While the depth depends on the delivered image quality, the implied
requirements are less stringent than the direct requirements on the image
quality variation over the field of view specified below.  The primary
purpose of these image depth requirements is to define allowed variation in
detector sensitivity.


\newpage
\paragraph{The Minimum Exposure Time\\}

\textbf{Specification:} The shortest possible exposure time will not be
longer than
\param{ETmin}{The minimum required exposure time (sec).}
seconds (Table~\ref{minTexp}).


\begin{table}[h]
\begin{tabular}{|r|r|r|r|}
\hline
Quantity   & Design Spec & Minimum Spec & Stretch Goal           \\
\hline
      ETmin (sec)  &      5       &      10      &       1       \\
\hline
\end{tabular}
\caption{The minimum exposure time (in seconds). }
\label{minTexp}
\end{table}

The minimum exposure time limits the ability to study fast temporal
changes in brightness and position. The required specification will
enable sampling of time scales three times shorter than the nominal
exposure time. As an added benefit, it will enable an extension of the
saturation limit $\sim$1 mag brighter than with the nominal exposures.
Note that the requirement on relative photometric accuracy specified in
Table \ref{relPhotometry} also applies to these shorter exposures.

%%%%%%%%%%%%%%
\subsubsection{        The Delivered Image Quality       }

The delivered image quality depends on atmospheric seeing and distortions
introduced by the system. It can be parametrized by the equivalent Gaussian
width (see below) and the ellipticity of the delivered PSF.  The deviations
of the image profile from the implied Gaussianity are parametrized by the
radii enclosing specified fractions of the total energy (light).

The weak lensing studies are particularly sensitive to the delivered image
quality (other science programs are only indirectly affected, \eg
through the dependence of the image depth on image size). As there is no
particular threshold to be achieved in the plausible 0.5--0.9 arcsec range,
the benefit is a monotonic function of improvements in delivered image
quality.


\paragraph{The delivered image size distribution\\}


The delivered image size, hereafter ``delivered seeing'' (as opposed to
atmospheric seeing), is expressed using the ``equivalent Gaussian width"
computed from
\begin{equation}
      \mathrm{seeing} = 0.663 \, \mathrm{pixelScale} \, \sqrt{n_{eff}} \,\,\mathrm{arcsec}.
\end{equation}
Here pixelScale is the pixel size in arcsec (0.2 for the baseline
design) and $n_{eff}$ is the effective number of pixels computed from
\begin{equation}
        n_{eff}  = {(\sum f_i)^2 \over \sum f_i^2},
\end{equation}
where $f_i$ is the image intensity (\ie the sum is over a bright
star).  In the limit of a perfect single Gaussian profile, the seeing computed
using these expressions is equal to the FWHM
($n_{eff}=2.27\,(\mathrm{seeing/pixelScale})^2$ for a single Gaussian). Note that
this approach is insensitive to the detailed image profile, and accounts
for the fact that atmospheric seeing cannot be described by a single
Gaussian at the required level of accuracy ($\sim$1\%).

%$n_{eff}$ is the same effective number of pixels that enters
%the computation of the image depth.

The image size is specified for three values of fiducial atmospheric
seeing: 0.44, 0.60 and 0.80 arcsec. These values are chosen as the three
quartiles of the seeing distribution measured at the Cerro Pach\'{o}n site
using DIMM at 500 nm, and corrected using an outer scale parameter of
30~m. The atmospheric seeing distribution at the
Cerro Pach\'{o}n site is illustrated in Appendix D.


\textbf{Specification:} The delivered seeing distribution across the field of
view will have a median not larger than
\param{S1}{The maximum delivered median seeing (arcsec).}
arcsec, with no more than
\param{SF1}{The maximum fraction of images with median seeing exceeding
S1$*$SX arcsec (see S1, SX) (\%).}
\% of images exceeding
\param{SX}{A scale factor on S1 used in defining SF1 (see S1, SF1).}
times S1 arcsec (Table~\ref{seeingT}).

\begin{table}[h]
\begin{tabular}{|l|r|r|r|}
\hline
        Quantity  &  Design Spec & Minimum Spec & Stretch Goal \\
\hline
       S1 (0.44)  &    0.56      &     0.59      &     0.53      \\
       S1 (0.60)  &    0.69      &     0.72      &     0.67      \\
       S1 (0.80)  &    0.87      &     0.89      &     0.85      \\
       SF1 (\%)   &    10        &      10       &       5       \\
          SX      &    1.1       &     1.2       &     1.1       \\
\hline
\end{tabular}
\caption{The delivered seeing distributions for three fiducial values
of atmospheric seeing (arcsec). These values apply to the $r$ and
$i$ bands.}
\label{seeingT}
\end{table}

The required image size is derived by assuming that the delivered image
quality will be dominated by atmospheric seeing effects and not by the
system. The design specification values reflect an error budget of 0.35
arcsec (for both telescope and camera, and including static and dynamic
errors), which is added in quadrature. The minimum specification and
stretch goal are computed using error budgets of 0.4 and 0.3 arcsec,
respectively. The system contribution to the delivered
image quality should not exceed 0.4 arcsec in any other band.

The above design specification for the image quality
requires that, for the median atmospheric seeing, the system
contribution to the delivered image quality never exceeds 15\%.
This requirement should be fulfilled irrespective of the airmass,
which limits the seeing degradation due to hardware away from the
zenith (e.g. due to gravity load).

\textbf{Specification:}
The allowed seeing error budget due to system at airmass=2 is
\param{SXE}{The allowed error budget due to system at airmass=2 (arcsec).}
arcsec (Table~\ref{SXEtable}).

\begin{table}[h]
\begin{tabular}{|l|r|r|r|}
\hline
        Quantity  &  Design Spec & Minimum Spec & Stretch Goal \\
\hline
            SXE      &        0.52       &        0.59           &        0.45     \\
\hline
\end{tabular}
\caption{The image quality error budget due to system at airmass=2 (arcsec).  \,\,\,\,\,\,\,\, \,\,\,\,\,\,\,\,\,\,\,}
\label{SXEtable}
\end{table}

Assuming that the atmospheric seeing increases with airmass, $X$, as $\propto X^{0.6}$,
the design specification for the allowed error budget due to the system (15\% degradation in image quality)
is defined at airmass of 2 and for the median seeing conditions (0.91 arcsec at $X=2$). The minimum specification
and the stretch goal are computed by scaling the design specification by 0.3/0.35 and 0.4/0.35,
respectively, in analogy with specifications for the S1 parameter (see Table~\ref{seeingT}).





\paragraph{The Image Sampling\\}

\textbf{Specification:} The pixel size will be smaller than
\param{pixSize}{The maximum pixel size (arcsec).} arcsec (Table~\ref{pixSize}).
\begin{table}[h]
\begin{tabular}{|l|r|r|r|}
\hline
        Quantity  & Specification \\
\hline
       pixSize     &        0.22      \\
\hline
\end{tabular}
\caption{The maximum pixel size (arcsec) to enable proper image sampling. \,\,\,\,\,\,\,\,\,\,\,\,\,\, \,\,\,\,\,\,\,\, }
\label{pixSize}
\end{table}

The pixel size must be smaller than the first quartile of the delivered image
size distribution  (0.56 arcsec) divided by 2.5. This coefficient is motivated by the
need to sample the point-spread function properly in the delivered images.




\vskip 0.2in \leftline \textbf{ The seeing spatial profile}

\textbf{Specification:} For a fiducial delivered seeing of 0.69 arcsec
(S1 from Table \ref{seeingT} for the median atmospheric seeing), at least 80\%
of the energy will be encircled within a radius of
\param{SR1}{The 80\% encircled energy diameter for median seeing (arcsec).}
arcsec, at least 95\% of the energy will be encircled within
\param{SR2}{The 90\% encircled energy diameter for median seeing (arcsec).}
arcsec, and at least 99\% of the energy will be encircled within
\param{SR3}{The 99\% encircled energy diameter for median seeing (arcsec).}
arcsec
(Table~\ref{Tspatprof}).

\begin{table}[h]
\begin{tabular}{|r|r|r|r|}
\hline
       Quantity    & Design Spec & Minimum Spec & Stretch Goal \\
\hline
      SR1 (arcsec) &    0.74       &   0.80       &   0.70        \\
      SR2 (arcsec) &    1.20       &   1.31       &   1.14        \\
      SR3 (arcsec) &    1.66       &   1.81       &   1.59        \\
\hline
\end{tabular}
\caption{The spatial profile (shape) for delivered seeing. These values
  apply to all the bands, as they are defined for a
  fiducial delivered seeing. For a different fiducial seeing, the
  SRx/seeing ratio, \ie the \textit{shape} of the delivered image,
  must be preserved.}
\label{Tspatprof}
\end{table}

The specified values were computed using a double-Gaussian profile that is
a good description of both typically-observed seeing profiles and that
expected for Kolmogorov turbulence
\begin{equation}
       p(x) = G(0,\sigma) + 0.1\, G(0,2\sigma),
\end{equation}
where $G(\mu,\sigma)$ is a two-dimensional Gaussian. For this profile,
$n_{eff}$ is 31\% larger than for a single Gaussian with the same FWHM. For
a seeing of 0.69 arcsec described by this profile, the radii enclosing
80\%, 95\% and 99\% of the energy are 0.67, 1.09, and 1.51 arcsec,
respectively. Note that it would be grossly incorrect to assume that the
seeing can be described by a single Gaussian, as the corresponding radii
are 0.52, 0.71, and 0.89 arcsec.

The design specifications were determined by multiplying the above radii by
1.1, and by 1.2 for the minimum specifications.  For the stretch goal
specifications, the multiplication factor is 1.05.  These requirements
limit the deviations from the above canonical profile, and in particular,
the amount of power in the expected power-law wings, due to the system. The
power-law wings observed for free atmospheric seeing have a much smaller
amplitude ($\sim$10 times, relative to the central intensity) than the
upper limit implied by the above requirements.


%%%%%%%% TT:


\paragraph{The image ellipticity distribution\\}

The image ellipticity is defined as $e ~~=~~ (\sigma^{2}_{maj} -
\sigma^{2}_{min}) / (\sigma^{2}_{maj} + \sigma^{2}_{min}) $, where
$\sigma^{2}_{maj}$ and $\sigma^{2}_{min}$ are, respectively, the
2$^{nd}$ moments of the best fit elliptical double Gaussian.  The best
fit elliptical Gaussian is used so as to minimize issues of truncation
of intensity sums in noisy data.  A double Gaussian is specified for consistency
with the image size specifications above (the two components are
assumed to have the same ellipticity and the same orientation).

\textbf{Specification:} For a delivered seeing of 0.69 arcsec, in a field
with a zenith distance of at most 10 degrees, the ellipticity
distribution across the field of view for unresolved sources will have
a median not larger than
\param{SE1}{The maximum median ellipticity across the field of view for
unresolved sources (ellipticity).},
with no more than
\param{EF1}{The maximum fraction of ellipticities exceeding an ellipticity
of SE2 (\%).}
\% of images exceeding the ellipticity of
\param{SE2}{EF1 \% of the ellipticities may exceed this value (see EF1) (ellipticity).}
(Table~\ref{TellipS}).

\begin{table}[h]
\begin{tabular}{|l|r|r|r|}
\hline
Quantity       & Design Spec & Minimum Spec & Stretch Goal \\
\hline
      SE1      &    0.04      &     0.05     &       0.03    \\
      EF1 (\%) &      5       &      10      &         5     \\
      SE2      &    0.07      &     0.1      &       0.05    \\
\hline
\end{tabular}
\caption{These values apply to the $r$ and $i$ bands.}
\label{TellipS}
\end{table}

These values are motivated by observations showing that the
ellipticity induced by atmospheric turbulence in a 10-second exposure
in 0.69 arcsec will be in the range 0.01-0.02.  The specification is
set such that the telescope+camera system does not contribute
appreciably to the
ellipticity beyond the natural limit set by the atmosphere (e.g. tracking
errors, jitter in the telescope).  This
specification does not by itself address weak lensing systematics,
because there are schemes for removing the influence of an anisotropic
PSF on the observed shapes of galaxies.  However, it is known that
these schemes leave smaller residuals if initially given isotropic
PSFs to begin with, hence the specification that the telescope+camera
not greatly degrade the natural limit. The overall image PSF ellipticity distribution
is further discussed in Section~\ref{sec:fullEllip} (see Table~\ref{fullEllip}).




%%%%%%%%%%%%%%
\subsubsection{Photometric Quality}
\label{photoacc}
The photometric accuracy is
specified through requirements on relative photometry (repeatability), the
system stability across the sky, color zero-points, and the transfer to a
physical flux scale (\ie calibration onto the AB magnitude scale).

A broad-band photometric system, such as LSST, aims to
deliver calibrated in-band flux
\begin{equation}
\label{Fb}
             F_b = \int{F_\nu(\lambda) \phi_b(\lambda) d\lambda},
\end{equation}
where $F_\nu(\lambda)$ is specific flux of an object \textit{at the top} of
the atmosphere and $\phi_b(\lambda)$ is the normalized system response
for the given band (the $\lambda^{-1}$  term reflects the fact that CCDs
are photon-counting devices)
\begin{equation}
\label{PhiDef}
\phi_b(\lambda) = {\lambda^{-1} S_b(\lambda) \over \int{\lambda^{-1} S_b(\lambda) d\lambda}}.
\end{equation}
Here, $S_b(\lambda)$ is the overall atmosphere + system throughput
\begin{equation}
\label{SDef}
         S_b(\lambda) = S_b^{sys}(\lambda) \times S_b^{atm}(\lambda).
\end{equation}
Traditionally, the in-band flux is reported on a magnitude scale,
and here we adopt AB magnitudes defined as
\begin{equation}
\label{ABmag}
             m_b = -2.5\log_{10}\left({F_b \over 3631 \, \mathrm{Jy}}\right).
\end{equation}

In order to interpret photometric measurements at the error level
specified below ($<$1\%), both $S_b^{sys}(\lambda)$ and $S_b^{atm}(\lambda)$
must be known with sufficient precision. Experience with precursor surveys,
such as SDSS, suggest that the dependence of both functions on
wavelength have to be directly measured to break the 1\% photometric error
barrier (especially for sources with complex spectral energy distributions,
such as supernovae). While the individual normalizations of $S_b^{sys}(\lambda)$ and
$S_b^{atm}(\lambda)$ are \textit{not} required (c.f. eq.~\ref{PhiDef}) to
interpret measurements using eq.~\ref{Fb}, the flux scale (calibration)
errors affect the reported values of $F_b$ (i.e. $m_b$). Therefore,
for each photometric measurement both $F_b$ and $\phi_b(\lambda)$
will be reported, together with their estimated uncertainties. These
two quantities will capture fundamental information included in LSST
measurements, and will enable accurate transformation of $F_b$
to systems with similar $\phi_b(\lambda)$ when the source spectral
energy distribution is known or assumed. In particular, corrections
to some standardized ``average'' LSST system, $\phi_b^{std}(\lambda)$,
will be defined during the commissioning period for the most relevant
spectral energy distributions (such as main sequence stars and normal
galaxies).




The requirements for photometric calibration accuracy are specified using
the following error decomposition (valid in the limit of small errors)
\begin{equation}
\label{photoSysErr}
 m_{cat} = m_{true} + \sigma + \delta_m(x,y,\phi,\alpha,\delta,SED,t) + \Delta_m,
\end{equation}
where $m_{true}$ is the true magnitude defined by eqs.~\ref{Fb} and
\ref{ABmag}, $m_{cat}$ is the cataloged LSST magnitude, $\sigma$ is the random
photometric error (including random calibration errors and count extraction
errors), and $\Delta_m$ is the overall (constant) offset of the
internal survey system from a perfect AB system (the six values of $\Delta_m$
are equal for \textit{all} the cataloged objects). Here, $\delta_m$ describes
the various systematic dependencies of the internal zeropoint error around
$\Delta_m$, such as position in the field of view ($x,y$), the normalized
system response ($\phi$), position on the sky ($\alpha, \delta$), and
the source spectral energy distribution ($SED$). Note that the average of
$\delta_m$ over the cataloged area is 0 by construction.

This error decomposition decouples ``internal absolute'' calibration
(i.e. producing an internally consistent system by minimizing $\delta_m$),
from that of ``external absolute'' calibrations (i.e. determining the six
$ugrizy$ $\Delta_m$  values for the LSST survey). Furthermore, the deviation
of the LSST system from a perfect AB system, $\Delta_m$, can be expressed
relative to a fiducial band, say $r$,
\begin{equation}
            \Delta_m = \Delta_r + \Delta_{mr}.
\end{equation}

The motivation for this separation is twofold. First, $\Delta_{mr}$ can
be constrained by considering the colors (spectral energy distributions)
of objects, \textit{independently from the overall flux scale} (this can
be done using both external observations and models). Second, there are
few science programs that crucially depend on knowing the ``gray
scale'' offset, $\Delta_r$, at the 1-2\% level. On the other hand,
knowing the ``band-to-band'' offsets, $\Delta_{mr}$, with such an
accuracy is \textit{critically important} for many applications (e.g.,
photometric redshifts of galaxies, type Ia supernovae cosmology, testing
of stellar and galaxy SED models).



\paragraph{The photometric repeatability (relative photometry)\\}

\textbf{Specification:} The rms of the unresolved source magnitude
distribution around the mean value (repeatability) will not exceed
\param{PA1}{The maximum rms of the unresolved source magnitude distribution
around the mean value (repeatability) (millimag).}
millimag (median distribution for a large number of sources). No more than
\param{PF1}{The maximum fraction of magnitudes deviating by more than PA2
from mean (\%).}
\% of the measurements will deviate by more than
\param{PA2}{At most PF1 \% of magnitudes may deviate by more than this from
the mean (millimag).} millimag from the mean (Table~\ref{relPhotometry}).


\begin{table}[h]
\begin{tabular}{|l|r|r|r|}
\hline
Quantity   & Design Spec & Minimum Spec & Stretch Goal   \\
\hline
      PA1 (millimag)  &     5       &       8      &        3  \\
      PF1  (\%)       &    10       &      20      &        5  \\
      PA2 (millimag)  &    15       &      15      &       10  \\
\hline
\end{tabular}
\caption{The specifications for photometric repeatability. The listed
values apply to the $g$, $r$ and $i$ bands.  The PA1 and PA2 values in
the $u$, $z$ and $y$ bands may be 50\% larger.}
\label{relPhotometry}
\end{table}

This requirement, defined for bright, unresolved sources (\ie those whose measurements
are not dominated by photon statistics, see \S~\ref{singleImageSpecs}), specifies the
distribution of random photometric errors, $\sigma$, and constrains both the repeatability
of extracting counts from images and the ability to monitor (or model) the changes in
normalized system response ($\phi$). In practice,
multiple visits will be used to compute the photometric rms for each individual star in each
bandpass and then study its dependence on position on the sky and on the camera, stellar
color, brightness, time, etc.

The specified values are driven by the photometric redshift accuracy,
the separation of stellar populations, detection of low-amplitude variable
objects (such as eclipsing planetary systems), and the search for
systematic effects in type Ia supernova light-curves. To verify that this
requirement is fulfilled, samples of predominantly non-variable stars will
have to be selected using appropriate color selection. Note that is
sufficient to have only two observations to diagnose that this requirement
is not fulfilled.


\paragraph{The spatial uniformity of photometric zeropoints\\}

\textbf{Specification:} The distribution width (rms) of the internal
photometric zero-point error (the system stability across the sky) will
not exceed
\param{PA3}{The maximum rms of the internal photometric zero-point error
(the system stability across the sky) (millimag).}
millimag, and no more than
\param{PF2}{The maximum fraction of internal photometric zero-point
errors exceeding PA4 (\%).}
\% of the distribution will exceed
\param{PA4}{At most PF2 \% of internal photometric zeropoint errors may
exceed this value (millimag).} millimag (Table~\ref{TphotomUniform}).

\begin{table}[h]
\begin{tabular}{|l|r|r|r|}
\hline
Quantity   & Design Spec & Minimum Spec & Stretch Goal   \\
\hline
      PA3 (millimag)  &    10     &       15    &        5     \\
      PF2  (\%)       &    10     &       20    &        5     \\
      PA4 (millimag)  &    15     &       20    &       15     \\
\hline
\end{tabular}
\caption{The specifications for the spatial uniformity of photometric
zeropoints. The values for PA3 and PA4 may be somewhat worse in the
$u$ band (but not by more than a factor of two).}
\label{TphotomUniform}
\end{table}

This requirement on the distribution of $\delta_m(x,y,\phi,\alpha,\delta,SED,t)$
primarily constrains the stability of the internal photometric system
across the sky.

The specified requirements are driven by the photometric redshift accuracy,
the separation of stellar populations, and the accuracy of inter-comparing
distance moduli from different supernovae. These requirements apply
to \textit{both} the bright and faint ends and constrain the non-linearity of the
flux scale. To verify that these requirements are
fulfilled, samples of standard stars may be needed (an alternative is to
track the shifts of morphological features in color-color diagrams).
Note that these requirements also place an upper limit on various systematic
errors, such as, for example, a correlation of internal photometric
zero-point error with the position on a sensor, and within the field of
view.


\paragraph{The band-to-band (flux ratio) photometric calibration\\}

\textbf{Specification:} The absolute band-to-band zero-point transformations
(color zero-points, \eg for constructing the spectral energy distribution,
SED) for main-sequence stars must be known with an accuracy of
\param{PA5}{Color zero-points for main-sequence stars must be known to
this accuracy (millimag).}
millimag (Table~\ref{TcolorOffsets}).

\begin{table}[h]
\begin{tabular}{|l|r|r|r|}
\hline
              Quantity   & Design Spec & Minimum Spec & Stretch Goal   \\
\hline
      PA5                &       5     &       10     &    3            \\
      PA5 (with $u$)     &      10     &       15     &    5            \\
\hline
\end{tabular}
\caption{The accuracy of color zero-points (in millimag). The second row
applies to colors constructed using the $u$ band photometry.}
\label{TcolorOffsets}
\end{table}

These requirements on $\Delta_{mr}$ are primarily driven by the desired accuracy of
photometric redshift estimates. Note that an overall stable gray error in
the absolute calibration of the system ($\Delta_r$) does not have an impact
on the above requirements. Such an error is specified next.



\paragraph{The overall external absolute photometry\\}

\textbf{Specification:} The LSST photometric system must transform to a
physical scale (\eg AB magnitude scale) with an accuracy of
\param{PA6}{The transformation from the LSST photometric system to a
physical scale must be at least this accurate (millimag).}
millimag (Table~\ref{ToffsetsAB}).

\begin{table}[h]
\begin{tabular}{|l|r|r|r|}
\hline
Quantity   & Design Spec  & Minimum Spec & Stretch Goal   \\
\hline
     PA6  &     10       &       20     &       5         \\
\hline
\end{tabular}
\caption{The accuracy of photometric system transformation to a physical scale
(in millimag). }
\label{ToffsetsAB}
\end{table}

The requirements on $\Delta_{r}$ are driven by the accuracy of absolute determination of
quantities such as luminosity and asteroid size for objects with well
determined distances. Note that the internal band-to-band transformations,
$\Delta_{mr}$, are required to be much more accurate as they may be calibrated
and controlled by other means, and are not sensitive to errors in overall
flux scale of photometric calibrators.



\paragraph{Further notes on photometry\\}


The photometry of resolved sources of moderate size (effective radius $\le$10
arcsec) must
have comparable quality (not worse than a factor of 2 in an rms sense, with the
same fraction of sources in the distribution tails, as per above requirements)
to unresolved stellar sources. The photometric measurements for resolved sources
(galaxies) have to include several standard magnitudes, such as Petrosian magnitudes,
as well as appropriate model magnitudes.

It is noteworthy that the technical aspects of how the required photometric precision
and accuracy will be achieved (\eg calibration schemes,
flat-field determination, corrections for atmospheric effects) are purposely
left out of this discussion as they belong to the Engineering Requirements
Documents; for example, an all-sky cloud camera or new software algorithms
(e.g. self-calibration, pioneered as \textit{\"{u}bercalibration} by SDSS)
may be needed to achieve the required photometric accuracy. The same
approach is taken when specifying astrometric requirements, described next.



%%%%%%%%%%%%%%%%%%%%%%%%%%%%%%%%%%%%
\subsubsection{Astrometric Quality}

The astrometric requirements are defined for bright unresolved sources (\ie
those whose measurements are not dominated by photon statistics, see
\S~\ref{singleImageSpecs}). The astrometric accuracy is specified through
requirements on relative astrometry (repeatability) and absolute
astrometry.


\paragraph{The relative astrometry\\}

\textbf{Specification:} The rms of the astrometric distance distribution for
stellar pairs with separation of D arcmin (repeatability) will not exceed
\param{AMx}{The maximum rms of the astrometric distance distribution for stellar
pairs with separations of D arcmin (repeatability) (x=1,2,3 $\rightarrow$ D=5,
20, 200 arcmin)\  (milliarcsec).}
milliarcsec (median distribution for a large number of
sources). No more than
\param{AFx}{The maximum fraction of astrometric distances which deviate by more than
ADx milliarcsec (see AMx) (\%).}
\% of the sample will deviate by more than
\param{ADx}{AFx of the astrometric distances will deviate by more than this (see AMx, AFx)
(milliarcsec).}
milliarcsec from the median. AMx, AFx,
and ADx are specified for D=5, 20 and 200 arcmin for x= 1, 2, and 3, in the
same order (Table~\ref{TastroRel}).

\begin{table}[h]
\begin{tabular}{|l|r|r|r|}
\hline
Quantity   & Design Spec & Minimum Spec & Stretch Goal     \\
\hline
    AM1 (milliarcsec)  &    10      &        20    &        5         \\
    AF1 (\%)          &    10      &        20    &        5         \\
    AD1 (milliarcsec)  &    20      &        40   &        10         \\
\hline
    AM2 (milliarcsec)  &    10      &        20    &        5         \\
    AF2 (\%)          &    10      &        20    &        5         \\
    AD2 (milliarcsec)  &    20      &        40   &        10         \\
\hline
    AM3 (milliarcsec)  &    15      &        30    &       10         \\
    AF3 (\%)          &    10      &        20    &        5         \\
    AD3 (milliarcsec)  &    30      &        50   &        20         \\
\hline
\end{tabular}
\caption{The specifications for astrometric precision. The three blocks
of values correspond to D=5, 20 and 200 arcmin, and
to astrometric measurements performed in the $r$ and $i$ bands.}
\label{TastroRel}
\end{table}

The three selected characteristic distances reflect the size of an
individual sensor, a
raft, and the camera. The required median astrometric precision is driven by
the desire to achieve a proper motion accuracy of 0.2 mas/yr and parallax
accuracy of 1.0 mas over the course of the survey. These two requirements
correspond to relative astrometric precision for a single image of 10 mas
(per coordinate).

About 25\% of blue stars ($g-r<1$) and 50\% of
red stars brighter than $r=20$ have proper motions greater than 10
mas/yr. Thus, to verify that these requirements are met, it may be
necessary to use quasars, which have a sky surface density of about 60 per
square degree for $r<20$ (implying a mean separation of $\sim$4 arcmin).

\textbf{Specification:} The astrometric reference frame for an image obtained
in a band other than $r$ will be mapped to the corresponding $r$ band image
such that the rms of the distance distribution between the positions on the
two frames will not exceed
\param{AB1}{The maximum rms distance between images in $r$ and in other bands (milliarcsec).}
milliarcsec (for a large number of bright sources). No more than
\param{ABF1}{The maximum fraction of rms distances between images in $r$ and in other bands
greater than AB2 (see AB1)(\%).}
\% of the measurements will deviate by more than
\param{AB2}{At most ABF1 \% of rms distances between images in $r$ and
in other bands will be greater than this value (milliarcsec).}
milliarcsec from the mean. The dependence of the distance between the
positions on the two frames on the source color and observing
conditions will be explicitly included in the astrometric model
(Table~\ref{Tastrob2b}).

\begin{table}[h]
\begin{tabular}{|l|r|r|r|}
\hline
Quantity   & Design Spec & Minimum Spec & Stretch Goal     \\
\hline
    AB1 (milliarcsec)  &    10      &        20    &        5         \\
    ABF1 (\%)          &    10      &        20    &        5         \\
    AB2 (milliarcsec)  &    20      &        40   &        10         \\
\hline

\end{tabular}
\caption{The requirements on the band-to-band astrometric transformation
accuracy (per coordinate in arbitrary band, relative to the $r$ band reference frame).}
\label{Tastrob2b}
\end{table}

The requirements on the band-to-band astrometric transformation accuracy
are driven by the detection of moving objects, de-blending of complex
sources, and astrometric accuracy for sources detected in a single-band
(\eg high-redshift quasars detected only in the $y$ band).


\paragraph{The absolute astrometry\\}

\textbf{Specification:} The LSST astrometric system must transform to an
external system (\eg ICRF extension) with the median accuracy of
\param{AA1}{The median accuracy of the astrometric transformation from
the LSST system to an external system (milliarcsec).}
milliarcsec (Table~\ref{TastroAbs}).

\begin{table}[h]
\begin{tabular}{|r|r|r|r|}
\hline
Quantity            & Design Spec & Minimum Spec & Stretch Goal     \\
\hline
  AA1 (milliarcsec)  &    50      &      100    &       20         \\
\hline
\end{tabular}
\caption{The median error in the absolute astrometric positions (per
coordinate, in milliarcsec).}
\label{TastroAbs}
\end{table}

The accuracy of absolute astrometry is driven by the linkage and orbital
computations for solar system objects. A somewhat weaker constraint is also
placed by the need for positional cross-correlation with external
catalogs. Note that the delivered absolute astrometric accuracy may be
fundamentally limited by the accuracy of astrometric standard catalogs.



\paragraph{The time-recording accuracy\\}

\textbf{Specification:}
The LSST system must record times (such as TAI)
with an internal (relative) accuracy of
\param{TACREL}{The internal (relative) time-recording accuracy (millisecond).}
milliseconds and an absolute accuracy of
\param{TACABS}{The absolute time-recording accuracy (millisecond).}
milliseconds (Table~\ref{TtimeRec}).

\begin{table}[h]
\begin{tabular}{|r|r|r|r|}
\hline
Quantity             & Design Spec & Minimum Spec & Stretch Goal     \\
\hline
 TACREL (millisec)   &     1  &    1   &    1 \\
 TACABS (millisec)   &    10  &   10   &    1 \\
\hline
\end{tabular}
\caption{The requirements for the time-recording accuracy.}
\label{TtimeRec}
\end{table}

The time-recording accuracy is driven by the linkage and
orbital computations for solar system objects. An assumed
nominal angular motion for a very fast object of 10 deg/day,
and a limit on astrometric accuracy of 5 milliarcsec, yield
a requirement of 10 millisec. The specification for internal
accuracy is adopted as ten times smaller than for absolute
accuracy.




\paragraph{Auxiliary System Characteristics\\}

The weak lensing science may be jeopardized by systematics in shape
measurements.
The effects of ghosting, which can also contribute to systematics in the
shape measurements, are partially addressed through photometric
requirements (\eg the fraction of objects with substandard photometry).
The presence of ghost images which can be confused with new point sources
and thus be a source of false transient detections must be minimized.
Further constraints, such as maximum sensor area loss and up/down time,
are addressed in the LSST System Requirements Document.

%%%%%%%%%%%%%%%%%%%%%%%%%%%%%%%%%%%%%%%%%%%%%%%%%%%%%%%%%%%%%%%%%%%%%%
%%%%%%%%%%%%%%%%%%%%%%%%%%%%%%%%%%%%%%%%%%%%%%%%%%%%%%%%%%%%%%%%%%%%%%
%%%%%%%%%%%%%%%%%%%%%%%%%%%%%%%%%%%%%%%%%%%%%%%%%%%%%%%%%%%%%%%%%%%%%%
%\newpage
\subsection{          The Full Survey Specifications                   }


By obtaining numerous images of the same area on the sky, LSST will be able
to significantly extend the scientific reach of the single images described
above.  The required total number of images depends on each particular science
program.  Studies of LSST science capabilities performed to date have
demonstrated that:
\begin{enumerate}
\item It is possible to design a ``universal cadence'' that would result in
a common database of observations to be used by all science programs (see
Appendix B).
\item The required survey lifetime is of the order 10 years. The strongest
constraints on this lower limit come from the required number of images to
perform robust weak lensing analysis and from the minimum time baseline to
obtain sufficiently accurate proper motion measurements. A somewhat less
quantitative constraint on the upper limit for the survey lifetime comes
from the desire to avoid ``stale science'' and to stay at the
technological forefront.
\end{enumerate}

As a result of these studies, the adopted baseline design (see Appendix A)
assumes a nominal 10-year duration with about 90\% of the observing
time allocated for the main LSST survey. The same assumption
was adopted here to derive the requirements described below.  Only visits
that satisfy the requirements listed in the previous section are counted
towards the specifications listed in this Section. For example, if the photometric
accuracy falls below requirements due to complex atmospheric cloud structure,
or due to extraneous noise sources inside the system, the data will not be counted.
The remaining 10\% of observing time will be used to obtain improved coverage
of parameter space such as very deep ($r\sim26$) observations, observations with
very short revisit times ($\sim$1 minute), and observations of  ``special'' regions
such as the Ecliptic, Galactic plane, and the Large and Small Magellanic Clouds.
A third type of survey, micro-surveys, that would use about 1\% of the time,
may also be considered.


Note that some quantities relevant for science analysis are indirectly
defined. For example, while the accuracy for photometric redshifts is not
specified, it is one of the main drivers for the bandpass selection, required photometric
accuracy per single visit, and the number of visits. Simulations show that
the requirements described here will result in photometric redshift
accuracy in the range 1-3\% (fractional rms for 1+z over the redshift range
0.2-3.0, the fraction of catastrophic failures is still being investigated;
also, a training sample of galaxies with spectroscopic redshift may be
needed in order to limit systematic errors). This performance is consistent
with the level required by the science drivers.
Unlike photometric redshifts whose accuracy is not very sensitive to
the distribution of visits in time,  the proper motion and parallax accuracies
can significantly deteriorate for some types of cadence strategies. Therefore,
the requirements for the proper motion and parallax accuracies  are explicitly
listed.



\paragraph{The sky area and distribution of visits vs. bandpass for the main survey\\}
\label{VvsB}

Detailed simulations show that, during its nominal 10-year  lifetime,
LSST will be capable of obtaining over 2,500,000 10 deg$^2$ visits
(using exposure time of 2x15 seconds).
Assuming an effective sky area of 20,000 deg$^2$ (less than the maximum
observable area from a given site because of airmass constraints), this
implies that each position on the sky can be visited about 1000 times
(ignoring field overlaps).  The distribution of these visits among the
bandpasses (filters) will have a direct impact on the science of the full
LSST survey.

The $r$ and $i$ bands should be preferentially selected over other bands
because they will be used for shape measurements. Other bands cannot
be neglected, however, because a broad wavelength coverage is required to achieve
desired photometric redshift accuracy for galaxies and needed color
information on transients such as supernovae. The impact of the visit
distribution across bands can be gauged from the following two cases. If
all 1000 visits were assigned equally to the $r$ and $i$ bands, their
final depth, assuming $\sqrt{N}$ scaling, would be 3.4 mag deeper than the
single image depth (see \S \ref{singleimagedepth}). If the visits were
distributed equally over 6 bands, then the final depth would be 2.8 mag
deeper than the single image depth. Note that the depth difference between
these two scenarios is only 0.6 mag.

Detailed simulations of LSST operations indicate that the following sky coverage and
allocation of observing time per band satisfies a variety of science drivers at a
nearly optimal level. However, \textit{the listed requirements should be
understood only as an illustration of the LSST capabilities because
the survey optimization is still in progress.}


\textbf{Specification:} The sky area uniformly covered by the main
survey will include
\param{Asky}{The sky area uniformly covered by the main survey.}
square degrees (Table~\ref{TskyArea}).

\begin{table}[h]
\begin{tabular}{|r|r|r|r|}
\hline
 Quantity            & Design Spec & Minimum Spec & Stretch Goal     \\
\hline
 Asky  (deg$^2$)   &  18,000      &  15,000       &  20,000        \\
\hline
\end{tabular}
\caption{The sky area uniformly covered by the main survey.}
\label{TskyArea}
\end{table}

The design specification for the sky area uniformly covered by the main
survey is a result of maximizing the number of galaxies detected at a
fiducial signal-to-noise ratio, given a fixed total exposure time.
The derived area corresponds to an airmass limit of 1.4 (including
accounting for the Galactic plane area and variable observing conditions),
which assures that the delivered image size degradation as a function of
airmass is not larger than the degradation induced by the system
(see Table~\ref{seeingT}), and that the image depth degradation due to
increased airmass is not larger than variations due to varying
sky brightness.



\textbf{Specification:} The sum of the median number of visits in each
band,
\param{Nv1}{The median of the distribution of the number of visits
across the sky in a given band.},
across the sky area specified in Table~\ref{TskyArea},
will not be smaller than Nv1 (Table~\ref{Nv1table}).
\begin{table}[h]
\begin{tabular}{|r|r|r|r|}
\hline
 Quantity            & Design Spec & Minimum Spec & Stretch Goal     \\
\hline
     Nv1             &   825   &   750     &    1000     \\
\hline
\end{tabular}
\caption{ The sum of the median number of visits in each
band across the sky area specified in Table~\ref{TskyArea}.}
\label{Nv1table}
\end{table}


\textit{An illustration}
of the distribution of Nv1 is shown in Table~\ref{TdepthCoadd}.
It is assumed that $\sim$10\% of the total observing time is allocated to
each of the $u$ and $g$ bands, and $\sim$20\% to each of the $rizy$ bands. These allocations
reflect the dependence of the desired final coadded depths on bandpass,
as well as a minimum number of visits in the $r$ and $i$ bands to enable
exquisite control of image shape systematics. The bandpass dependence of
the coadded depths is optimized using a mean galaxy spectral energy
distribution at a redshift of $\sim$2. The adopted
depths are also well matched to spectral energy distributions of blue stars
(main sequence stars will enable kinematic and
metallicity studies all the way to the presumed edge of the Milky Way halo)
and low-redshift quasars, and to those of very red objects (e.g., high-redshift
quasars and galaxies, the coldest stars, highly reddened stars in the Milky
Way disk). The simulations assume an exposure time of 30 seconds per visit,
which is a result of the survey optimization using \textit{simultaneous}
constraints for the single visit depth, the number of visits, the revisit
time, and the surveying efficiency (see Section 2.2 in the LSST overview
paper, \citep{2008arXiv0805.2366I}).


\begin{table}[h]
\begin{tabular}{|l|r|r|r|r|r|r|}
\hline
   Quantity           &   u   &   g    &   r   &   i   &   z   &  y   \\
\hline
 Nv1 (design spec.)   &  56 (2.2) & 80 (2.4) &  184 (2.8) & 184 (2.8) & 160 (2.8) & 160 (2.8) \\
\hline
Idealized Depth         &    26.1   &    27.4  &     27.5   &  26.8     &   26.1    & 24.9  \\
\hline
\end{tabular}
\caption{An \textit{illustration} of the distribution of the number of visits as
a function of bandpass, obtained by detailed simulations of LSST operations
that include realistic weather, seeing and sky brightness distributions,
as well as allocation of about 10\% of the
total observing time to special programs. The median
number of visits per field for all bands is 824. For convenience, the numbers in
parentheses show the corresponding gain in depth (magnitudes), assuming
$\sqrt{N}$ scaling. The last row shows the total \textit{idealized} coadded
depth for the design specification median depth of a single image (assuming
5$\sigma$ depths at $X=1$ of $u=23.9$, $g=25.0$, $r=24.7$, $i=24.0$, $z=23.3$ and
$y=22.1$, from Table~\ref{ImageDBtable}), and the above design specification
for the total number of visits.  The coadded image depth losses due to airmass
greater than unity are not taken into account. For a large suite of simulated main survey
cadences, they are about 0.2-0.3 mag, with the median airmass in the range 1.2-1.3.
}
\label{TdepthCoadd}
\end{table}

The coadded image depth losses due to airmass greater than unity are not
taken into account because they depend on cadence details. For a large suite
of simulated main survey cadences, they are about 0.2-0.3 mag, with the median airmass in
the range 1.2-1.3.
Table~\ref{TdepthCoadd} specifies the \textit{median} number of visits. The actual
number of visits may vary across the sky. For example, the regions close to
the Ecliptic may benefit from longer exposures to improve the size and distance
limit for distant solar system objects. These regions may also require a
larger than median number of visits to improve
the completeness level for small asteroids. On the other hand, the Galactic
plane regions may have smaller than median number of visits (\eg very deep
images may be confusion limited). Details of such variations are not specified
here, with an understanding that \textit{the adopted observing strategy will not
jeopardize the goals of any of the four main science themes} (\eg the full
avoidance of the Galactic plane would have a severe impact on the Galactic
structure studies). For the same reason, minimum specification
and stretch goal values per band are also not specified. The median total number of visits
(824 here) is expected to be uncertain by about 10\% due to unpredictable
observing conditions and details of the final adopted observing strategy.



\paragraph{Distribution of visits in time\\}

The LSST will be capable of observing 20,000 deg$^2$ of the sky in two
bands every three nights. While the data obtained with such a cadence will
contribute greatly to studies of optical transients, it is desirable to
explore much shorter scales, down to 1 minute. This can be achieved with
frequent revisits to the same field, or by utilizing field overlap.  The
details of the revisit time distribution, and its dependence on the covered
area, will greatly depend on the adopted observing strategy and here only
rough guidance is provided.

\textbf{Specification:} At least
\param{RVA1}{The minimum area of sky with multiple observations separated by nearly
uniformly sampled time scales ranging from 40 sec to 30 min (square degrees).}
square degrees will have multiple observations separated by nearly uniformly
sampled time scales ranging from 40 sec to 30 min (Table~\ref{TfastArea}).

\begin{table}[h]
\begin{tabular}{|r|r|r|r|}
\hline
Quantity            & Design Spec & Minimum Spec & Stretch Goal     \\
\hline
  RVA1  (deg$^2$)   &  2,000      &  1,000       &     3,000        \\
\hline
\end{tabular}
\caption{The minimum area with fast (40 sec -- 30 min) revisits.}
\label{TfastArea}
\end{table}

The requirements are derived using the universal cadence described in
Appendix B as a minimalistic scenario. It shows that $\sim$10\% of the
total area can be revisited on short time scales by utilizing field
overlaps.

Strong constraints on the distribution of visits in time come from the goal
of accurately measuring stellar parallax and proper motion (see Section~\ref{sec:MW}).
Irrespective of the delivered astrometric precision, parallax measurements will
not be of sufficient accuracy if the majority of visits are clustered around
the same epoch. Similarly, proper motion measurements require that a large
fraction of the observations are spread over as long a baseline as possible.

\textbf{Specification:} The median trigonometric parallax accuracy across the main
survey area for unresolved sources with $r=24$ must be at least
\param{SIGpara}{The trigonometric parallax error for a $r$=24 source (mas).} mas.
The median proper motion accuracy per coordinate across the main
survey area for such sources must be at least
\param{SIGpm}{The proper motion error for a $r$=24 source (mas/yr).}
The median trigonometric parallax accuracy across the main survey area for unresolved
sources detected only in the $y$ band (at 10$\sigma$) must be at least
\param{SIGparaRed}{The trigonometric parallax error for 10$\sigma$
$y$-band only detections (mas).} mas.
(Table~\ref{TablePARAPM}).

\begin{table}[h]
\begin{tabular}{|r|r|r|r|}
\hline
Quantity                & Design Spec & Minimum Spec & Stretch Goal       \\
\hline
  SIGpara (mas)             &    3.0     &     6.0          &    1.5         \\
  SIGpm   (mas/yr)        &    1.0     &     2.0          &    0.5         \\
  SIGparaRed (mas)        &   6.0     &    10.0          &    3.0         \\
\hline
\end{tabular}
\caption{The required trigonometric parallax and proper motion accuracy.}
\label{TablePARAPM}
\end{table}

These requirements constrain the distribution of visits in time and are derived
from the requirements for proper motion accuracy of 0.2 mas/yr at $r=20.5$,
trigonometric parallax accuracy of 1 mas at $r=22.4$, and trigonometric parallax
accuracy of 6 mas for red sources with only 10$\sigma$ $y$-band detections, per
Section~\ref{sec:MW}. Detailed simulations show that the baseline cadence can deliver
this performance for main sequence stars (Section 3.3.3 in the LSST overview paper,
\citep{2008arXiv0805.2366I}.), if the astrometric requirements from the previous section are met.


Additional constraints on the distribution of visits in time come from
the requirement to efficiently detect and characterize moving Solar System
objects. In addition to their own science drivers, this requirement is driven
by the fact that they may significantly contaminate samples of transients
(see Section~\ref{sec:transients}).  In order to enable robust inter-night
linking of their detections, the current baseline cadence assumes two
visits per night  (see Appendix B).


\paragraph{Distribution of visits vs. observing conditions\\}

Observing conditions include, but are not limited to, seeing, sky
brightness (affected by lunar phase and time of night), transparency, and
airmass. Designing a cadence which will optimize performance on the mixed
science goals described in this document will necessarily involve some
compromises between conflicting goals. The algorithm employed by the
observation scheduler will balance these goals, take advantage of current
conditions, and maintain as uniform coverage as possible in both time and
location on the sky.

It is assumed that the observing strategy will follow standard practices as
much as possible when selecting the bandpass and sky location of a
particular visit (\eg blue exposures should preferentially take
place around the new moon, and bright time observations will avoid the moon
as much as possible). However, for some transient science, especially high-redshift
supernovae, it is critical to sample the $z$ and $y$ bands throughout the
light-curve, and this will necessitate altering these standard practices to
some extent. Weak lensing science mandates that the $r$ and $i$ band
observations be performed in the best seeing nights, and at low airmass. It
also requires a large range of angles between the pupil, detector, and sky
to minimize possible systematic errors.



%%%%%% TT


\paragraph{Galaxy shear measurement accuracy, and PSF ellipticity residuals\\}
\label{sec:fullEllip}

The important angular scales for weak lensing two-point shear correlation probes of dark
energy are 10 arcminute to several degrees, where the cosmic shear correlation signal can
get as small as $10^{-6}$ at low redshift and several degree scales. The hemisphere sky
coverage of LSST is needed in order to achieve the required statistical precision in these
shear correlations, and to suppress cosmic variance.  As explained in section~\ref{sec:DE},
for the LSST ``gold'' sample of 4 billion galaxies (defined by $i<25.3$), the resulting random
component of the shear cross correlation noise level is about $3\times10^{-7}$ over this
angular range up to several degrees.  It is thus important that the systematic component be
less than about 30\% of this noise (to become negligible when added in quadrature).  The
requirement then is that the galaxy shear extraction algorithm (and system hardware) be
capable of delivering this level of galaxy shear systematics residual.  Because of the dominant
galaxy shape shot noise, the shear errors in a large sample are dominated by PSF errors at the
galaxy positions (together with errors in model fitting given the PSF). This then leads to a
requirement on the residual PSF ellipticity correlations on these angular scales.

A limit to the effectiveness of the PSF-correction schemes is our
knowledge of the delivered PSF within each image, which is sampled
at high Galactic latitude roughly three times per square arcminute by a
high S/N ratio star and must be interpolated at the positions of the
galaxies. (There is a color dependence, such that the stellar PSF must
also be interpolated to the colors of the galaxies, but we do not
address that issue here.)

To address these systematics, we first define ellipticity components
\begin{equation}
 e_1 = {\sigma_1^2 - \sigma_2^2 \over \sigma_1^2 + \sigma_2^2},
\end{equation}
and
\begin{equation}
 e_2 = {2 \sigma_{12}^2 \over \sigma_1^2 + \sigma_2^2},
\end{equation}
where $\sigma_1^2$ and $\sigma_2^2$ are the 2nd moments of the stellar
image along some set of perpendicular axes, and $\sigma_{12}^2$ is the
covariance (again, for the best-fit elliptical Gaussian).
A PCA fit to the ellipticity components’ dependence on the position within
each CCD is made, and we examine the residuals.
It is the \textit{correlation of these residuals} $\delta e_1$
and $\delta e_2$, not their
mean size, which sets the level of weak lensing systematics.  We
define the residual ellipticity auto and cross correlation functions
\begin{equation}
 E_1(\theta) = \, \langle \delta e_1^{(i)}\delta e_1^{(j)} \rangle,
\end{equation}
\begin{equation}
 E_2(\theta) = \, \langle \delta e_2^{(i)}\delta e_2^{(j)} \rangle,
\end{equation}
and
\begin{equation}
 E_X(\theta) = \, \langle \delta e_1^{(i)}\delta e_2^{(j)} \rangle,
\end{equation}
where angle brackets indicate averaging over all pairs of stars $i$ and
$j$ at a given angular separation $\theta$.  Again, we use the natural
limit of the atmosphere as a guide.  Observations indicate that the residuals $E_1$,
$E_2$, and $E_X$ are $\sim 10^{-4}$ at scales of an arcminute and smaller for
a 10-second exposure at 0.7 arcsec delivered seeing, falling below shot noise
levels at $\theta = 5$ arcmin.  It is a requirement that LSST images not degrade
this quality significantly. The defocus spectrum from atmosphere induced seeing,
combined with optics astigmatism and FPA (focal plane array) non-flatness, produces
so-called ``B mode" shear (non-zero $E_X$). Using as priors the measured FPA non-flatness, the wavefront
curvature measurements for that image, and the optics astigmatism vs. defocus
data,
one can optimally fit the PSF over the image. Similarly, requirements on these
instrument parameters can be deduced from the science requirements on the
residual ellipticity correlations.



\textbf{Specification:} Using the full survey data, the $E_1$, $E_2$, and
$E_X$ residual PSF ellipticity correlations
averaged over an arbitrary FOV must have the median less than
\param{TE1}
{The maximum value for the median ellipticity correlation function
on $\le$1 arcmin scale for unresolved bright sources using the full survey data
(hereafter ellipticity).}
for $\theta \le 1$ arcmin, and less than
\param{TE2}
{The maximum value for the median ellipticity correlation function
on $\ge$5 arcmin scale for unresolved bright sources using the full survey data
(hereafter ellipticity).}
for $\theta \ge 5$ arcmin. No more than
\param{TEF}
{The maximum fraction of ellipticities for unresolved bright sources using the
full survey data exceeding TE3 or TE4 (\%).}
\% of images will have these medians larger than
\param{TE3}{At most TEF \% of ellipticities for unresolved bright sources
using the full survey data may exceed this value on $\le$1 arcmin scales
(see TE1).}
for $\theta \le 1$ arcmin, and
\param{TE4}{At most TEF \% of ellipticities for unresolved bright sources
using the full survey data may exceed this value on $\ge$5 arcmin scales
(see TE3).}
for $\theta \ge 5$ arcmin (Table~\ref{fullEllip}).

\begin{table}[h]
\begin{tabular}{|l|r|r|r|}
\hline
Quantity       &   Design Spec & Minimum Spec  & Stretch Goal \\
\hline
      TE1 & $2 \times 10^{-5}$  & $3 \times 10^{-5}$ & $1 \times 10^{-5}$ \\
      TE2 & $1 \times 10^{-7}$  & $3 \times 10^{-7}$ & $5 \times 10^{-8}$ \\
      TEF & $15\%$              & $15\%$             & $10\%$             \\
      TE3 & $4 \times 10^{-5}$  & $6 \times 10^{-5}$ & $2 \times 10^{-5}$ \\
      TE4 & $2 \times 10^{-7}$  & $5 \times 10^{-7}$ & $1 \times 10^{-7}$ \\
\hline
\end{tabular}
\caption{These residual PSF ellipticity correlations apply to the $r$ and $i$ bands.}
\label{fullEllip}
\end{table}

The residual ellipticity correlations vary smoothly so it is sufficient to specify
limits in these two angular ranges. On 1 arcmin to 5 arcmin scales, these residual
ellipticity correlations put LSST systematics a factor of a few below the
weak lensing shot noise, i.e. statistical errors will dominate over systematics.
On larger scales, the noise level imposed by nature due to shot noise plus
cosmic variance is almost scale-independent, whereas the atmospheric
contribution to systematics becomes negligible.  Therefore the
specifications on 5 arcmin scales apply to all larger scales as well
(as per section \ref{WLstudies}).  On scales larger than the field of view, sources of
systematic error have less to do with the instrumentation than with the operations
(due to the seeing distribution), software, and algorithms.  It is recommended
that the scheduler attempt to match the seeing distribution in each
patch of sky, so that at the end of the survey at most a small fraction
of patches will have substantially better or worse seeing than average.



\subsection{Data Processing and Management Requirements}

Detailed requirements on data processing and management will be described
in the LSST System Requirements Document (for example, specifications for catalog completeness
and reliability). Here, only a rough guidance is provided. There will be three
main categories of data products:

\begin{itemize}
\item \textbf{Level 1} data products are generated continuously every observing night,
including alerts to objects that have changed brightness or position.
\item \textbf{Level 2} data products will be made available as annual Data Releases
and will include images and measurements of positions, fluxes, and shapes, as well
as variability information such as orbital parameters for moving objects and
an appropriate compact description of light curves.
\item \textbf{Level 3} data products will be created by the community, including project teams,
using suitable Applications Programming Interfaces (APIs) that will be provided
by the LSST Data Management System. The Data Management System will also provide
at least 10\% of its total capability for user-dedicated processing and user-dedicated storage.
The key aspect of these capabilities is that they will reside ``next to" the LSST data,
avoiding the latency associated with downloads. They will also allow the science
teams to use the database infrastructure to store their results.
\end{itemize}


One of the most fundamental advantages of LSST survey is the use of individual
images and corresponding modeling of varying observing conditions (such as seeing
and the background emission) in order to reduce systematic errors in measured
parameters.  Many types of systematic errors (those that do not decrease with
exposure time or source brightness; e.g., errors in photometry due to imperfect
point-spread-function modeling or due to flatfield errors) that will be present in
individual visits will be uncorrelated between different observations (for example,
seeing will be uncorrelated; the data taking strategy will include dithering). These
uncorrelated systematic errors can be significantly reduced by appropriate processing
of data that utilizes the full knowledge of varying observing conditions. Substantial
benefits of such error reduction for LSST science,  together with a general principle
that the measurement errors for fundamental quantities should not be dominated
by algorithmic performance, lead to the following requirements for the image
processing software.

\textbf{Minimum Specification:} observing conditions, such as point-spread function and
background level, will be evaluated and known for each visit, and for all objects
detected in deep coadded data and in difference images. The object properties,
such as positions, fluxes, and shapes, will be measured for all these objects in
each visit. These measurements must meet the design specifications from this
document; more complex issues, such as deblending of overlapping sources
and galaxy photometry, will be addressed with the best existing algorithms.

\textbf{Design Specification:} a flexible and robust software framework that can process
data using methods which preserve information from individual exposures, and
use it to constrain best-fit model parameters, will be developed.
This framework will not assume, nor be dependent on, a specific
set of processing algorithms. The algorithms to be ultimately deployed must model
point-spread function, basic extended source parameters (such as galaxy shapes),
source blending, the impact of observing conditions and other effects, at least at
the level required to meet the SRD design specifications for the relevant parameters.

\textbf{Stretch goal:} a set of improved algorithms will be developed, meeting a general principle,
specified at the end of section~\ref{sec:defs} in this document, that the measurement errors for
fundamental quantities, such as astrometry, photometry and image size, should not be dominated
by algorithmic performance. In other words, whenever LSST data
will support a better performance than required by the SRD design specifications,
various algorithmic improvements which will deliver such performance will be
undertaken within the project resource boundaries. These algorithmic improvements
will be prioritized by the project, guided by input from the
LSST Science Council and the LSST Science Collaborations.

\textbf{Specification:} Images and catalog data will be released to
publicly-accessible repositories as soon as possible after they are
obtained. This latency, and the exact form of the data to be continuously
released, are left unspecified at this time pending further discussion
within the project. Data on likely optical transients, however, will be
released with a latency of at most
\param{OTT1}{The minimum latency for releasing data on optical transients (minutes).}
minutes.

Acknowledging that science thrives on repeatability of results, however, it
is recognized that specific, fixed ``snapshots'' of the data should
periodically be released so that data used in published analyses can
unambiguously be referenced. The object catalogs in these snapshot data
releases will include an extensive list of measured properties that will
allow a variety of science analyses without the need to reprocess
images. These catalogs and images, corrected for instrumental artifacts,
and photometrically and astrometrically calibrated, will be released to
public every
\param{DRT1}{The maximum interval between public releases of ``snapshot''
catalog and image data (years).}
years (Table~\ref{TDMcadence}). The catalogs and images will be released
for both single visits and for appropriately co-added data (such as those
optimized for depth and for weak lensing analysis). The catalog data in
these releases may be more extensive (\ie reflect more analysis)
than that released on a continuing basis. The catalogs will be released in
a format that will allow efficient data access and analysis (such as a
database and query system).

\begin{table}[h]
\begin{tabular}{|r|r|r|r|}
\hline
     Quantity         &   Design Spec & Minimum Spec  & Stretch Goal   \\
\hline
    DRT1 (year)       &       1.0     &      2.0      &       0.5     \\
    OTT1 (min)        &       1.0     &      2.0      &       0.5     \\
\hline
\end{tabular}
\caption{Requirements for the data release cadence and for the transient
reporting latency.}
\label{TDMcadence}
\end{table}

The requirement on the data release cadence is a compromise between ``too
often'', which may have an impact on the system's efficiency, and ``too
slow'', which may have an impact on the system's science outcome and its
perception within the community.  The requirement on the transient
reporting latency is essentially driven by the total exposure time, which
sets an intrinsic scale for the time resolution of transient sources.

The fast release of data on likely optical transients will include
measurements of position, flux, size and shape, using appropriate
weighting functions, for all the objects detected above
\param{transSNR}{The minimum signal-to-noise ratio in difference
image for reporting detection of a transient object}
signal-to-noise ratio in difference images (design specification: 5).
The data stream will also include prior variability information and
data from the same night, if available. The prior variability
information will at the very least include low-order light-curve moments
and probability that the object is variable, and ideally the full
light curves in all available bands.

\textbf{Specification:} The system should be capable of reporting such data for
at least  \param{transN}{The minimum number of candidate transients per field
of view that the system can report in real time}  candidate transients per
field of view and visit (Table~\ref{transN}).
\begin{table}[h]
\begin{tabular}{|r|r|r|r|}
\hline
     Quantity         &   Design Spec & Minimum Spec  & Stretch Goal   \\
\hline
    transN              &       10$^4$    &      10$^3$      &       10$^5$     \\
\hline
\end{tabular}
\caption{The minimum number of candidate transients per field
of view that the system can report in real time.}
\label{transN}
\end{table}

The users
will have an option of a query-like pre-filtering of this data stream
in order to select likely candidates for specific transient type. Users
may also query the LSST science database at any time for additional
information that may be useful, such as the properties of static objects
that are positionally close to the candidate transients.
Several pre-defined filters optimized for traditionally popular transients,
such as supernovae and microlensed sources, will also be available,
as well as the ability to add new pre-defined filters as the survey
continues.


\subsection{Further Improvements and Changes}

A number of LSST-related design and scientific studies are under way that
may affect the requirements described in this document. In addition, it is
quite possible that further specifications may be requested during the
process of distilling this document into the LSST Engineering Requirements
Document. Any such changes to this document will need to be brought to the
attention of the LSST Science Council, who will review the case and, if
appropriate, propose changes to the LSST Change Control Board. Please report
such, and any other, concerns and comments to \v{Z}eljko
Ivezi\'{c}\footnote{E-mail: ivezic@astro.washington.edu}.

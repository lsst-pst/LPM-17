\section{Introduction}

\B{Three recent nationally endorsed reports by the National Academy of Sciences
concluded that a dedicated wide-field imaging telescope with an effective aperture
of 6--8 meters is a high priority for US planetary science, astronomy, and physics
over the next decade. The Large Synoptic Survey Telescope (LSST) described here is
such a system.}

\R{The main purpose of this document is to define science-driven requirements for the
data products to be delivered by the Large Synoptic Survey Telescope (LSST). }
The LSST is envisioned to be a large, wide-field ground based telescope
designed to obtain sequential images covering over half the sky every few nights.
The current baseline design would allow to do so in two photometric bands every three
nights. This baseline design (for details see Appendix A) involves a 3-mirror system with an
8.4 m primary mirror, which feeds three refractive correcting elements inside a camera,
providing a 10 deg$^2$ field of view sampled by a 3 Gigapixel focal plane array.
The total effective system throughput (\'etendue) is expected to be greater than
300 m$^2$ deg$^2$,
which is more than an order of magnitude larger than that of any existing facility.
The survey will yield contiguous overlapping imaging of $\sim$20,000 square degrees
of sky in six optical bands covering the wavelength range 320--1050 nm.
Detailed simulations that include measured weather statistics and a variety
of other effects which affect observations predict that each sky location can be
visited about 100 times per year, with 30 sec exposure time per visit.

The range of scientific investigations which would be enabled by such a
dramatic improvement in survey capability is extremely broad and
\R{is summarized in detail in the LSST Science Book\footnote{Available from
http://www.lsst.org/lsst/scibook}.}  However, \B{at this stage of the program
it would not be} \G{it is not} feasible to make an exhaustive study of the scientific requirements
appropriate to all of them. \G{To define quantitative science drivers and resulting requirements,}
we therefore limit our attention in this document to four
main science themes:
\begin{enumerate}
\item Constraining Dark Energy and Dark Matter
\item Taking an Inventory of the Solar System
\item Exploring the Transient Optical Sky
\item Mapping the Milky Way
\end{enumerate}

Each of these four themes itself encompasses a variety of analyses, with
varying sensitivity to instrumental and system parameters.  It is our belief
that the analyses encompassed by our four science themes
fully exercise the technical capabilities of the system,
such as photometric and astrometric accuracy and image quality.  The working
paradigm at this time is that all such investigations will utilize a common
database constructed from an optimized observing program. An example of
such a program is described in Appendix B.

Below, we include short summaries of the science goals in each of these four
theme areas and the assumptions that have been invoked in translating these
into the minimum and design specification parameters.  This document concludes
with Tables of Science Requirements, in which we have integrated the
constraints from the different programs.

These science requirements are made in the context of what we forecast for
the scientific landscape in \B{2012} \R{2015}. Clearly science will not stand still in the
intervening time.   Using current plans for smaller surveys and precursor projects
one may calculate efficiencies and gauge the likely progress on a number of
LSST-related scientific frontiers. Some advances in each area will be made,
but the LSST remains the ultimate facility for each key area covered in this SRD.
Indeed, LSST represents such a large leap in throughput and survey capability
that in these key areas the LSST remains uniquely capable of addressing sharply
these fundamental questions about our universe.

